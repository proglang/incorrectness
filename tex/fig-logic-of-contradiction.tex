\begin{mathpar}
  \inferrule[Id]{}{ \FormA \Entails \FormA }

  \inferrule[Thin]{
    \FormA \Entails \SuccD}{
    \FormA \Entails \SuccD, \SuccC
  }

  \inferrule[Cut]{
    \FormA \Entails \SuccD, \FormB \\
    \FormB \Entails \SuccC
  }{
    \FormA \Entails \SuccD, \SuccC
  }
  \\
  \inferrule[$\Conj$L1]{
    \FormA \Entails \SuccD 
  }{
    \FormA \Conj \FormB \Entails \SuccD
  }

  \inferrule[$\Conj$L2]{
    \FormB \Entails \SuccD 
  }{
    \FormA \Conj \FormB \Entails \SuccD
  }

  \inferrule[$\Conj$R]{
    \FormA \Entails \SuccD, \FormB \\
    \FormA \Entails \SuccD, \FormC
  }{
    \FormA \Entails \SuccD, \FormB \Conj \FormC
  }
  \\
  \inferrule[$\vee$L]{
    \FormA \Entails \SuccD \\
    \FormB \Entails \SuccD
  }{
    \FormA \vee \FormB \Entails \SuccD
  }

  \inferrule[$\vee$R]{
    \FormA \Entails \SuccD, \FormB, \FormC
  }{
    \FormA \Entails \SuccD, \FormB \vee \FormC
  }
  \\
  \inferrule[$\butnot$L]{
    \FormA \Entails \SuccD, \FormB
  }{
    \FormA \butnot \FormB \Entails \SuccD
  }

  \inferrule{
    \FormA \butnot \FormB \Entails \SuccD    
  }{
    \FormA \Entails \SuccD, \FormB
  }

  \inferrule[$\Truth$R]{}{ \FormA \Entails \Truth }
\end{mathpar}

%%% Local Variables:
%%% mode: latex
%%% TeX-master: "ityping"
%%% End:
